\subsection{Implemented solution}

The suggested solution is a packet parser analyzing the flow and content of packets
spanned out between the different purdue levels. The parser is used as a supplement
to the exisiting ids solution to eliminate the need for reconfiguration. 
The parser is implemented using Python and controls the points selected in Section \ref{sec:feature_selection}.
If the parser detects any inconsitencies in traffic flow between levels or system states 
a notification is sent to the IDS. In that manner the IDS or SIEM remains the centralized 
decision unit.

The pcap files are read into the parser using the Pyshark package. An filter is
applied to filter out everything but the OPC-UA pakets. The opcua traffic is then 
flitered further to only contain read responses. These are the messages containing 
state values. The state values are moved into a dataframe containing one column 
for each state and one row for each connection between server and client. 

This process is performed on packets from each SPAN. The result is two pcap set objects and
two dataframes. The dataframes are compared, and if there is a mismatch, the state is
logged and a notification sent to the decision making unit. Further on the dataframes
are analyzed for state irregularities. These are the conditions presented in \ref{sec:feature_selection} which are 
only dependent on the flare system and the condition "BDV open while PRV" closed as this in
a simple mannner can be matched towards the state of the ESD system. The specific
state checks are therefore:

\begin{itemize}
 \item BDV open and PRV closed, check ESD status
 \item BDV open while RP/drain closed and level high
 \item Liquid filling but not large gas flow
\end{itemize}

The pcap set object containing all the opcua traffic is also analyzed for 
mismatches. Firstly the length of both captures are controlled. A mismatch
in length would indicate that packets have been dropped or supressed. Then all packets are
counted for each connection between the client and server. A mismatch would indicate that
the polling or setting of a state has been dropped or suppressed. 

The method does not utilize any statistical methods, which are
the ones used for machine learning. Instead, exact matching is used to 
perform several of the checks. As mentioned in Section \ref{sec:ids_theory}, this
is often used in signature-based detection. The result is lower number of false positives, 
which is a requirement when working with safety-critical systems. 

\subsubsection{Limitations}

- The parser uses currently uses files to perform detection, not live monitoring.
- time-consuming method. Batch-collecting packets creates delays, 
    but as long as they are no more than 4 minutes, is that ok?
- correlate the correct packets between levels. do they have an id? time to live 
    sequence number?

TODO WEDNESDAY 3.5
- prepare presentation
- test if the collection can be done with two servers and one client
- collect the level 1/2 datasets - daisychain the pollings
- perform the possible tests